\documentclass{article}
\usepackage{hyperref}
\usepackage[linguistics]{forest}
\usepackage{graphicx}
\usepackage[margin=1in]{geometry}
\usepackage{adjustbox}
\usepackage{bookmark}
\usepackage{rotating}
\usepackage{amsmath, bm, amssymb}

\title{Homework \#1\\ 
    \vspace{0.5cm}
    \normalsize John Karasev\\
    \vspace{0.2cm}
    \normalsize STAT 672\\
    \vspace{0.2cm}
    January 22, 2020 }
\date{}
\begin{document}
\maketitle

\section{Orthogonal projection in a hilbert space}
\[\pi_V(f) \in V;\]
\[\langle f - \pi_V(f), g \rangle =0 \ \forall g \in V; \]
\begin{enumerate}
    \item   
        \begin{equation}
        \begin{aligned}
        \|f\|^2 &= \langle f, f \rangle \\
        & = \langle \pi_v(f) + (f - \pi_v(f)), \pi_v(f) + (f - \pi_v(f)) \rangle  \notag \\
        & = \langle \pi_v(f), \pi_v(f) \rangle + 2\langle \pi_v(f), f - \pi_v(f) \rangle + \langle f - \pi_v(f), f - \pi_v(f) \rangle \notag \\
        & = \|\pi_v(f)\|^2 + \|f - \pi_v(f)\|^2\ center\ term\ cancels\ since\ orthogonal 
        \end{aligned}
        \end{equation}
    \item 
        \begin{equation}
        \begin{aligned}
        \langle f, \pi_v(g) \rangle &= \langle (f-\pi_v(f)) + \pi_v(f), \pi_v(g) \rangle \\
        & = \langle f-\pi_v(f) , \pi_v(g) \rangle +  \langle \pi_v(f) , \pi_v(g) \rangle \notag \\
        & = \langle \pi_v(f) , \pi_v(g) \rangle\ by\ orthogonality
        \end{aligned}
        \end{equation} 
        
        \begin{equation}
        \begin{aligned}
        \langle g, \pi_v(f) \rangle &= \langle (g-\pi_v(g)) + \pi_v(g), \pi_v(f) \rangle \\
        & = \langle g-\pi_v(g), \pi_v(f) \rangle +  \langle \pi_v(g) , \pi_v(f) \rangle \notag \\
        & = \langle \pi_v(g) , \pi_v(f) \rangle\ by\ orthogonality
        \end{aligned}
        \end{equation}

        \[ \langle g, \pi_v(f) \rangle = \langle f, \pi_v(g) \rangle\]
    \item 
\end{enumerate}


\section{A reproducing kernel is positive definite}
\begin{enumerate}
    \item 
        \[K(x,y) = \langle K_x(.), K_y(.) \rangle\ by\ reproducing\ property\]
        \[K(y,x) = \langle K_y(.), K_x(.) \rangle\ by\ reproducing\ property\] 
        \[K(x,y) = \langle K_x(.), K_y(.) \rangle = \langle K_y(.), K_x(.) \rangle\ =  K(y,x)\]
    \item
        \begin{equation}
        \begin{aligned}
        for\ any\ n \in \mathbb{N},\ (a_1, ..., a_n) \in \mathbb{R}^n\ ,\ \\
        (x_1, ..., x_n) \in \mathbb{X}^n \\
        \sum_{i,j=1}^{n} a_i a_j K(x_i, x_j) &= \sum_{i,j=1}^{n} a_i a_j \langle Kx_i, Kx_j\rangle_{\mathcal{H}} \\
        & = \bigg \| \sum_{i=1}^{n} a_i Kx_i \bigg \|_{\mathcal{H}} \ge 0
        \end{aligned}
        \end{equation}
    \item 
        \begin{equation}
        \begin{aligned}
        |\langle f, K_x \rangle|& \le  \|f\| \| K_x \|\ true\ by\ Cauchy-Schwartz\ inequality \notag \\
        |\langle f, K_x \rangle|& \le  \|f\| \sqrt{\langle K_x, K_x \rangle} \ by\ norm\ definition\notag \\
        |f(x)| & \le \|f\|\sqrt{K(x,x)}\ reproducing\ property
        \end{aligned}
        \end{equation}
\end{enumerate}

\section{RKHS over a finite set}
\begin{enumerate}
    \item 
        
        \[\langle \bm{f}, \bm{g} \rangle = \bm{f}^T\bm{K}^{-1}\bm{g}\quad
        \bm{f},\bm{g} \in \mathbb{R}^{2m+1}\]

        \begin{equation}
            \boldsymbol{\mathcal{N}}(\bm{x};\bm{\mu},\bm{\Sigma} ) 
            = \frac{1}{(2\pi)^\frac{n}{2}|\bm{\Sigma}|^\frac{1}{2}}
            exp\bigg\{\frac{-(\bm{x}-\bm{\mu})^T\bm{\Sigma^}{-1}(\bm{x}-\bm{\mu})}{2} \bigg\} \notag
            \quad \bm{x} \in \mathbb{R}^{n}
        \end{equation}
        
        
        \begin{equation}
            \begin{aligned}
                p(\bm{f}) &= Ce^{-\frac{\|\bm{f}\|^{2}_H}{2}} \\
                &= C exp \bigg\{ -\frac{\bm{f}^T\bm{K}^{-1}\bm{f}}{2} \bigg\} \notag \\
                &= \frac{1}{(2\pi)^\frac{2m+1}{2}|\bm{K}|^\frac{1}{2}}
                   exp \bigg\{ -\frac{\bm{f}^T\bm{K}^{-1}\bm{f}}{2} \bigg\} \notag \\
                p(\bm{f}) &= \boldsymbol{\mathcal{N}}(\bm{f}; \bm{0}, \bm{K})
            \end{aligned}
        \end{equation}
        \begin{equation}
            \begin{aligned}
                \therefore\ \bm{\mu} &= \bm{0} \notag \\
                \bm{\Sigma} &= \bm{K}
            \end{aligned} 
        \end{equation}

        \item $C$ is a normalizing constant so the probability distribution sums to 1
            \[C = \frac{1}{(2\pi)^\frac{2m+1}{2}|\bm{K}|^\frac{1}{2}}\]
            which is taken from the $\boldsymbol{\mathcal{N}}(\bm{f}; \bm{0}, \bm{K})$ 
            normalization constant
        \item 
            \begin{enumerate}
                \item 
                    kernel matrix: $\bm{K} = [k_{ij}] = e^{-\frac{(i-j)^2}{2\tau^2}}$ \\
                    \vspace{0.5cm}
                    \adjustbox{valign=t}{\includegraphics[width=14cm]{../figures/gausst0_5.pdf}}
                    \vspace{0.5cm}
                \item functions become smoother as $\tau$ increases \\
                \vspace{0.5cm}
                \adjustbox{valign=t}{\includegraphics[width=14cm]{../figures/gausst0_15.pdf}}
                \vspace{0.5cm}

                \vspace{0.5cm}
                \adjustbox{valign=t}{\includegraphics[width=14cm]{../figures/gausst0_35.pdf}}
                \vspace{0.5cm}

                \vspace{0.5cm}
                \adjustbox{valign=t}{\includegraphics[width=14cm]{../figures/gausst0_7.pdf}}
                \vspace{0.5cm}

                \vspace{0.5cm}
                \adjustbox{valign=t}{\includegraphics[width=14cm]{../figures/gausst1_5.pdf}}
                \vspace{0.5cm}

                \vspace{0.5cm}
                \adjustbox{valign=t}{\includegraphics[width=14cm]{../figures/gausst2_5.pdf}}
                \vspace{0.5cm}

                \vspace{0.5cm}
                \adjustbox{valign=t}{\includegraphics[width=14cm]{../figures/gausst4.pdf}}
                \vspace{0.5cm}

                \vspace{0.5cm}
                \adjustbox{valign=t}{\includegraphics[width=14cm]{../figures/gausst10.pdf}}
                \vspace{0.5cm}
                \item
                \vspace{0.5cm}
                \adjustbox{valign=t}{\includegraphics[width=14cm]{../figures/polyd2.pdf}}
                \vspace{0.5cm}

                \vspace{0.5cm}
                \adjustbox{valign=t}{\includegraphics[width=14cm]{../figures/tanh1.pdf}}
                \vspace{0.5cm}

                \vspace{0.5cm}
                \adjustbox{valign=t}{\includegraphics[width=14cm]{../figures/tanh5.pdf}}
                \vspace{0.5cm}

                \vspace{0.5cm}
                \adjustbox{valign=t}{\includegraphics[width=14cm]{../figures/min.pdf}}
                \vspace{0.5cm}

                \vspace{0.5cm}
                \adjustbox{valign=t}{\includegraphics[width=14cm]{../figures/linear.pdf}}
                \vspace{0.5cm}
            \end{enumerate}
\end{enumerate}




\end{document}
